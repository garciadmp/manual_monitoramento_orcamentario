% Options for packages loaded elsewhere
\PassOptionsToPackage{unicode}{hyperref}
\PassOptionsToPackage{hyphens}{url}
\PassOptionsToPackage{dvipsnames,svgnames,x11names}{xcolor}
%
\documentclass[
  letterpaper,
]{report}

\usepackage{amsmath,amssymb}
\usepackage{iftex}
\ifPDFTeX
  \usepackage[T1]{fontenc}
  \usepackage[utf8]{inputenc}
  \usepackage{textcomp} % provide euro and other symbols
\else % if luatex or xetex
  \usepackage{unicode-math}
  \defaultfontfeatures{Scale=MatchLowercase}
  \defaultfontfeatures[\rmfamily]{Ligatures=TeX,Scale=1}
\fi
\usepackage{lmodern}
\ifPDFTeX\else  
    % xetex/luatex font selection
\fi
% Use upquote if available, for straight quotes in verbatim environments
\IfFileExists{upquote.sty}{\usepackage{upquote}}{}
\IfFileExists{microtype.sty}{% use microtype if available
  \usepackage[]{microtype}
  \UseMicrotypeSet[protrusion]{basicmath} % disable protrusion for tt fonts
}{}
\makeatletter
\@ifundefined{KOMAClassName}{% if non-KOMA class
  \IfFileExists{parskip.sty}{%
    \usepackage{parskip}
  }{% else
    \setlength{\parindent}{0pt}
    \setlength{\parskip}{6pt plus 2pt minus 1pt}}
}{% if KOMA class
  \KOMAoptions{parskip=half}}
\makeatother
\usepackage{xcolor}
\setlength{\emergencystretch}{3em} % prevent overfull lines
\setcounter{secnumdepth}{5}
% Make \paragraph and \subparagraph free-standing
\ifx\paragraph\undefined\else
  \let\oldparagraph\paragraph
  \renewcommand{\paragraph}[1]{\oldparagraph{#1}\mbox{}}
\fi
\ifx\subparagraph\undefined\else
  \let\oldsubparagraph\subparagraph
  \renewcommand{\subparagraph}[1]{\oldsubparagraph{#1}\mbox{}}
\fi


\providecommand{\tightlist}{%
  \setlength{\itemsep}{0pt}\setlength{\parskip}{0pt}}\usepackage{longtable,booktabs,array}
\usepackage{calc} % for calculating minipage widths
% Correct order of tables after \paragraph or \subparagraph
\usepackage{etoolbox}
\makeatletter
\patchcmd\longtable{\par}{\if@noskipsec\mbox{}\fi\par}{}{}
\makeatother
% Allow footnotes in longtable head/foot
\IfFileExists{footnotehyper.sty}{\usepackage{footnotehyper}}{\usepackage{footnote}}
\makesavenoteenv{longtable}
\usepackage{graphicx}
\makeatletter
\def\maxwidth{\ifdim\Gin@nat@width>\linewidth\linewidth\else\Gin@nat@width\fi}
\def\maxheight{\ifdim\Gin@nat@height>\textheight\textheight\else\Gin@nat@height\fi}
\makeatother
% Scale images if necessary, so that they will not overflow the page
% margins by default, and it is still possible to overwrite the defaults
% using explicit options in \includegraphics[width, height, ...]{}
\setkeys{Gin}{width=\maxwidth,height=\maxheight,keepaspectratio}
% Set default figure placement to htbp
\makeatletter
\def\fps@figure{htbp}
\makeatother

\makeatletter
\@ifpackageloaded{tcolorbox}{}{\usepackage[skins,breakable]{tcolorbox}}
\@ifpackageloaded{fontawesome5}{}{\usepackage{fontawesome5}}
\definecolor{quarto-callout-color}{HTML}{909090}
\definecolor{quarto-callout-note-color}{HTML}{0758E5}
\definecolor{quarto-callout-important-color}{HTML}{CC1914}
\definecolor{quarto-callout-warning-color}{HTML}{EB9113}
\definecolor{quarto-callout-tip-color}{HTML}{00A047}
\definecolor{quarto-callout-caution-color}{HTML}{FC5300}
\definecolor{quarto-callout-color-frame}{HTML}{acacac}
\definecolor{quarto-callout-note-color-frame}{HTML}{4582ec}
\definecolor{quarto-callout-important-color-frame}{HTML}{d9534f}
\definecolor{quarto-callout-warning-color-frame}{HTML}{f0ad4e}
\definecolor{quarto-callout-tip-color-frame}{HTML}{02b875}
\definecolor{quarto-callout-caution-color-frame}{HTML}{fd7e14}
\makeatother
\makeatletter
\makeatother
\makeatletter
\@ifpackageloaded{bookmark}{}{\usepackage{bookmark}}
\makeatother
\makeatletter
\@ifpackageloaded{caption}{}{\usepackage{caption}}
\AtBeginDocument{%
\ifdefined\contentsname
  \renewcommand*\contentsname{Table of contents}
\else
  \newcommand\contentsname{Table of contents}
\fi
\ifdefined\listfigurename
  \renewcommand*\listfigurename{List of Figures}
\else
  \newcommand\listfigurename{List of Figures}
\fi
\ifdefined\listtablename
  \renewcommand*\listtablename{List of Tables}
\else
  \newcommand\listtablename{List of Tables}
\fi
\ifdefined\figurename
  \renewcommand*\figurename{Figure}
\else
  \newcommand\figurename{Figure}
\fi
\ifdefined\tablename
  \renewcommand*\tablename{Table}
\else
  \newcommand\tablename{Table}
\fi
}
\@ifpackageloaded{float}{}{\usepackage{float}}
\floatstyle{ruled}
\@ifundefined{c@chapter}{\newfloat{codelisting}{h}{lop}}{\newfloat{codelisting}{h}{lop}[chapter]}
\floatname{codelisting}{Listing}
\newcommand*\listoflistings{\listof{codelisting}{List of Listings}}
\makeatother
\makeatletter
\@ifpackageloaded{caption}{}{\usepackage{caption}}
\@ifpackageloaded{subcaption}{}{\usepackage{subcaption}}
\makeatother
\makeatletter
\@ifpackageloaded{tcolorbox}{}{\usepackage[skins,breakable]{tcolorbox}}
\makeatother
\makeatletter
\@ifundefined{shadecolor}{\definecolor{shadecolor}{rgb}{.97, .97, .97}}
\makeatother
\makeatletter
\makeatother
\makeatletter
\makeatother
\ifLuaTeX
  \usepackage{selnolig}  % disable illegal ligatures
\fi
\IfFileExists{bookmark.sty}{\usepackage{bookmark}}{\usepackage{hyperref}}
\IfFileExists{xurl.sty}{\usepackage{xurl}}{} % add URL line breaks if available
\urlstyle{same} % disable monospaced font for URLs
\hypersetup{
  pdftitle={Manual de Monitoramento Orçamentário do Programa de Metas 2025-2028},
  pdfauthor={(SEPLAN/SIME/CPMA)},
  colorlinks=true,
  linkcolor={blue},
  filecolor={Maroon},
  citecolor={Blue},
  urlcolor={Blue},
  pdfcreator={LaTeX via pandoc}}

\title{Manual de Monitoramento Orçamentário do Programa de Metas
2025-2028}
\author{\textbf{(SEPLAN/SIME/CPMA)}}
\date{}

\begin{document}
\maketitle
\ifdefined\Shaded\renewenvironment{Shaded}{\begin{tcolorbox}[enhanced, frame hidden, interior hidden, sharp corners, borderline west={3pt}{0pt}{shadecolor}, boxrule=0pt, breakable]}{\end{tcolorbox}}\fi

\renewcommand*\contentsname{Table of contents}
{
\hypersetup{linkcolor=}
\setcounter{tocdepth}{2}
\tableofcontents
}
\bookmarksetup{startatroot}

\hypertarget{boas-vindas}{%
\chapter{Boas vindas}\label{boas-vindas}}

É com muita satisfação que lhe convidamos a navegar por este
\textbf{Manual de Monitoramento Orçamentário do Programa de Metas
2025-2028}! O nosso objetivo com este material é apoiar você - que
representa o seu órgão na \textbf{Rede SMAE} - a prestar informações
sobre o planejamento e a execução orçamentária de cada meta no sistema
oficial de monitoramento da Prefeitura de São Paulo - o SMAE. Esperamos
que o conteúdo deste documento seja útil não apenas para fornecer um
guia operacional sobre como abastecer o SMAE, mas também para reforçar
objetivos comuns entre a sua pasta e a Secretaria Municipal de
Planejamento e Eficiência (SEPLAN), que visam enfrentar as seguintes
questões:

\begin{itemize}
\tightlist
\item
  Como entregar equipamentos, programas e serviços à população
  garantindo uma relação custo-benefício adequada?
\item
  Como reduzir a dependência de movimentações orçamentárias ao longo do
  exercício?
\item
  Como evitar que recuros disponíveis fiquem sem reserva, recursos
  reservados fiquem sem empenhos e recursos empenhados fiquem sem
  liquidação?
\end{itemize}

A contribuição da Rede SMAE e de SEPLAN para enfrentar as questões
mencionadas reside em dar um passo pequeno, porém essecial: saber como a
Prefeitura de São Paulo emprega recursos no Programa de Metas. Vamos
entender melhor como monitorar o planejamento e a execução orçamentário
do PdM?

\bookmarksetup{startatroot}

\hypertarget{o-que-uxe9-o-monitoramento-oruxe7amentuxe1rio-do-programa-de-metas}{%
\chapter{O que é o monitoramento orçamentário do Programa de
Metas?}\label{o-que-uxe9-o-monitoramento-oruxe7amentuxe1rio-do-programa-de-metas}}

O monitoramento orçamentário é o processo contínuo de acompanhamento e
avaliação da execução orçamentária do Programa de Metas (PdM) da
Prefeitura de São Paulo (PMSP). \textbf{Mas para quê monitorar?}

Existem três razões principais:

\begin{longtable}[]{@{}
  >{\raggedright\arraybackslash}p{(\columnwidth - 2\tabcolsep) * \real{0.2083}}
  >{\raggedright\arraybackslash}p{(\columnwidth - 2\tabcolsep) * \real{0.7917}}@{}}
\toprule\noalign{}
\endhead
\bottomrule\noalign{}
\endlastfoot
\textbf{Dimensão} & \textbf{Descrição} \\
\textbf{Gerencial} & Garantir que os recursos orçamentários previstos
sejam utilizados de forma eficiente e em consonância com as prioridades
estabelecidas pela gestão municipal, possibilitando ajustes de rota
sempre que necessário. \\
\textbf{Governamental} & Garantir que as peças orçamentárias (Plano
Plurianual - PPA; Lei de Diretrizes Orçamentárias - LDO e Lei do
Orçameno Anual - LOA) compreendam as prioridades estabelecidas pela
gestão municipal, possibilitando coerência administrativa e capacidade
de planejamento e execução. \\
\textbf{Transparência} & Informar a sociedade civil e os órgãos de
controle, cumprindo as prerrogativas do
\href{https://legislacao.prefeitura.sp.gov.br/leis/lei-0-de-04-de-abril-de-1990}{Art.
69-A, da Lei Orgânica do Município} e do
\href{https://legislacao.prefeitura.sp.gov.br/leis/decreto-63336-de-10-de-abril-de-2024}{Decreto
63.336/2024}. \\
\end{longtable}

Sabemos que a tarefa de prover informações orçamentárias não é a mais
simples. É, na verdade, muito trabalhosa. Mas por entendermos que ela
cumpre funções em diferentes dimensões - seja para subsidiar decisões
do(a) Prefeito(a), seja para fomentar a transparência e o controle
social -, queremos tornar esta tarefa tanto menos árdua quanto mais
valiosa.

Vamos, então, falar sobre quem faz o quê no monitoramento orçamentário
do Programa de Metas?

\bookmarksetup{startatroot}

\hypertarget{quem-faz-o-monitoramento-oruxe7amentuxe1rio-do-programa-de-metas}{%
\chapter{Quem faz o monitoramento orçamentário do Programa de
Metas?}\label{quem-faz-o-monitoramento-oruxe7amentuxe1rio-do-programa-de-metas}}

Uma vez que a Secretaria Municipal de Planejamento e Eficiência (SEPLAN)
é responsável tanto pelo Programa de Metas quanto pela elaboração das
peças orçamentárias, além de outras atribuições, é fundamental conhecer
as diferentes áreas internas, bem como os pontos de interlocução com a
sua pasta. Vamos apresentar na tabela, a seguir, um conjunto de
interações que SEPLAN estabelece com a sua secretaria para além do
Programa de Metas, para, em seguida, voltarmos ao objetivo deste Manual.

\begin{longtable}[]{@{}
  >{\raggedright\arraybackslash}p{(\columnwidth - 4\tabcolsep) * \real{0.0736}}
  >{\raggedright\arraybackslash}p{(\columnwidth - 4\tabcolsep) * \real{0.7057}}
  >{\raggedright\arraybackslash}p{(\columnwidth - 4\tabcolsep) * \real{0.2174}}@{}}
\toprule\noalign{}
\begin{minipage}[b]{\linewidth}\raggedright
Unidade de SEPLAN
\end{minipage} & \begin{minipage}[b]{\linewidth}\raggedright
O que faz?
\end{minipage} & \begin{minipage}[b]{\linewidth}\raggedright
Interlocutor no órgão
\end{minipage} \\
\midrule\noalign{}
\endhead
\bottomrule\noalign{}
\endlastfoot
SEPLAN/SIME/CPMA & \begin{minipage}[t]{\linewidth}\raggedright
\begin{enumerate}
\def\labelenumi{\arabic{enumi}.}
\tightlist
\item
  Monitora a execução física e orçamentária do Programa de Metas, por
  meio do SMAE.
\item
  Gerencia o Observasampa e coleta indicadores a serem atualizados na
  plataforma.
\item
  Subsidia a utilização do Módulo de Planos Setoriais do SMAE.
\end{enumerate}
\end{minipage} & \begin{minipage}[t]{\linewidth}\raggedright
\begin{enumerate}
\def\labelenumi{\arabic{enumi}.}
\tightlist
\item
  Rede SMAE.
\item
  Representante do Grupo Técnico de Indicadores.
\item
  Rede SMAE.
\end{enumerate}
\end{minipage} \\
SEPLAN/COPOM & \begin{minipage}[t]{\linewidth}\raggedright
\begin{enumerate}
\def\labelenumi{\arabic{enumi}.}
\tightlist
\item
  Elabora os projetos de lei do PPA, LDO e LOA.
\end{enumerate}
\end{minipage} & \begin{minipage}[t]{\linewidth}\raggedright
\begin{enumerate}
\def\labelenumi{\arabic{enumi}.}
\tightlist
\item
  Grupos de planejamento orçamentário.
\end{enumerate}
\end{minipage} \\
SEPLAN/COPOM/DIPROJ & \begin{minipage}[t]{\linewidth}\raggedright
\begin{enumerate}
\def\labelenumi{\arabic{enumi}.}
\tightlist
\item
  Monitora as principais fontes de pressão orçamentária ao longo da
  execução orçamentária anual, sob a perspectiva do orçamento geral de
  cada órgão (e não apenas do Programa de Metas).
\end{enumerate}
\end{minipage} & \begin{minipage}[t]{\linewidth}\raggedright
\begin{enumerate}
\def\labelenumi{\arabic{enumi}.}
\tightlist
\item
  Unidades orçamentárias-financeiras das pastas e Rede SMAE.
\end{enumerate}
\end{minipage} \\
SEPLAN/COPOM/DIPAR & \begin{minipage}[t]{\linewidth}\raggedright
\begin{enumerate}
\def\labelenumi{\arabic{enumi}.}
\tightlist
\item
  Conduz e monitora os processos e compromissos do Orçamento Cidadão.
\end{enumerate}
\end{minipage} & \begin{minipage}[t]{\linewidth}\raggedright
\begin{enumerate}
\def\labelenumi{\arabic{enumi}.}
\tightlist
\item
  Grupos de planejamento orçamentário e Rede SMAE.
\end{enumerate}
\end{minipage} \\
SEPLAN/COPOM/CGO & \begin{minipage}[t]{\linewidth}\raggedright
\begin{enumerate}
\def\labelenumi{\arabic{enumi}.}
\tightlist
\item
  Promove a analise, o julgamento e as respectivas movimentações
  orçamentárias propostas pelas pastas (as movimentações que envolvem
  ações do Programa de Metas são analisadas, também, por
  SEPLAN/SIME/CPMA).
\end{enumerate}
\end{minipage} & \begin{minipage}[t]{\linewidth}\raggedright
\begin{enumerate}
\def\labelenumi{\arabic{enumi}.}
\tightlist
\item
  Unidades orçamentárias-financeiras das pastas.
\end{enumerate}
\end{minipage} \\
SEPLAN/SIME/CAE & \begin{minipage}[t]{\linewidth}\raggedright
\begin{enumerate}
\def\labelenumi{\arabic{enumi}.}
\tightlist
\item
  Monitora projetos da carteira prioritária do Prefeito (parte dos
  projetos fazem parte do Programa de Metas).
\item
  Monitora obras (parte das obras fazem parte do Programa de Metas).
\end{enumerate}
\end{minipage} & \begin{minipage}[t]{\linewidth}\raggedright
\begin{enumerate}
\def\labelenumi{\arabic{enumi}.}
\tightlist
\item
  Gerentes de projetos, indicados pelos Gabinetes.
\item
  Interlocutores indicados pelos Gabinetes.
\end{enumerate}
\end{minipage} \\
SEPLAN/SIME/CFOU & \begin{minipage}[t]{\linewidth}\raggedright
\begin{enumerate}
\def\labelenumi{\arabic{enumi}.}
\tightlist
\item
  Monitora fundos e operações urbanas (parte dos planos de aplicação dos
  fundos e das operações urbanas fazem parte do Programa de Metas).
\end{enumerate}
\end{minipage} & \begin{minipage}[t]{\linewidth}\raggedright
\begin{enumerate}
\def\labelenumi{\arabic{enumi}.}
\tightlist
\item
  Interlocutres indicados pelos Gabinetes.
\end{enumerate}
\end{minipage} \\
\end{longtable}

Na \textbf{Gestão 2025-2028}, sete unidades de SEPLAN podem manter
comunicações com a sua pasta, visando levantar informações necessárias
ao cumprimento das atribuições previstas no
\href{https://legislacao.prefeitura.sp.gov.br/leis/decreto-64341-de-2-de-julho-de-2025}{Decreto
64.341/2025}, que define a atuação institucional de SEPLAN.

\begin{tcolorbox}[enhanced jigsaw, titlerule=0mm, rightrule=.15mm, toprule=.15mm, breakable, leftrule=.75mm, left=2mm, colbacktitle=quarto-callout-tip-color!10!white, colframe=quarto-callout-tip-color-frame, toptitle=1mm, bottomtitle=1mm, coltitle=black, title=\textcolor{quarto-callout-tip-color}{\faLightbulb}\hspace{0.5em}{Tip}, colback=white, opacitybacktitle=0.6, arc=.35mm, bottomrule=.15mm, opacityback=0]

\textbf{Uma vez que as interações podem ser realizadas por
interlocutores diferentes, tanto em SEPLAN (CPMA, COPOM, CAE etc.)
quanto na sua pasta (Rede SMAE, Grupo de Planejamento Orçamentário
etc.), é fundamental identificar os(as) profissonais que exercem este
papel.}

\end{tcolorbox}

Como pode ser observado na tabela, o monitoramento orçamentário do
Programa de Metas é conduzido pela Coordenadoria de Planejamento,
Monitoramento e Avaliação, da Secretaria Executiva de Monitoramento e
Informações Estratégicas, da SEPLAN (SEPLAN/SIME/CPMA). Esta é a unidade
responsável por:

\begin{itemize}
\item
  Estabelecer diretrizes e orientações aos órgãos sobre \textbf{O QUE
  DEVE SER CONTABILIZADO PARA FINS DE MONITORAMENTO DO PROGRAMA DE
  METAS};
\item
  Organizar as \textbf{ETAPAS DE MONITORAMENTO ORÇAMENTÁRIO DO PROGRAMA
  DE METAS};
\item
  Oferecer \textbf{SUPORTE TÉCNICO E DIFUNDIR O USO DO SMAE}.
\end{itemize}

Vamos entender como cada uma desses pontos funciona na prática?

\bookmarksetup{startatroot}

\hypertarget{o-que-deve-ser-contabilizado-para-fins-de-monitoramento-oruxe7amentuxe1rio-do-programa-de-metas}{%
\chapter{O que deve ser contabilizado para fins de monitoramento
orçamentário do Programa de
Metas?}\label{o-que-deve-ser-contabilizado-para-fins-de-monitoramento-oruxe7amentuxe1rio-do-programa-de-metas}}

São consideradas despesas do Programas de Metas:

\begin{itemize}
\item
  Despesas com projetos técnicos, pré-projetos e estudos preliminares
  para implantação de obras previstas no Programa de Metas;
\item
  Despesas com desapropriações;
\item
  Despesas para construção de novos equipamentos ou realização de novas
  obras, tanto em dotações de projetos quanto em dotações de atividades;
\item
  Despesas para gerenciamento (físico e, quando aplicável, social) de
  novos equipamentos ou novas obras;
\item
  Despesas para reformas, ampliações ou requalificações em equipamentos;
\item
  Despesas para o custeio de equipamentos, programas ou serviços
  implementados;
\item
  Despesas de exercícios anteriores (DEA), desde que restritos aos itens
  anteriores.
\end{itemize}

\begin{tcolorbox}[enhanced jigsaw, titlerule=0mm, rightrule=.15mm, toprule=.15mm, breakable, leftrule=.75mm, left=2mm, colbacktitle=quarto-callout-note-color!10!white, colframe=quarto-callout-note-color-frame, toptitle=1mm, bottomtitle=1mm, coltitle=black, title=\textcolor{quarto-callout-note-color}{\faInfo}\hspace{0.5em}{Note}, colback=white, opacitybacktitle=0.6, arc=.35mm, bottomrule=.15mm, opacityback=0]

\textbf{Exemplos}

A Secretaria da Educação (SME) está realizando obras via SP-OBRAS para
inaugurar uma novo Centro Educacional Unificado (CEU) em agosto de 2026:
antes da inauguração, deve-se contabilizar os valores com projetos,
estudos, obras, desapropriações (se houver). Após a inauguração, deve-se
contabilizar: despesas com segurança, contas de consumo, pessoal (apenas
se tiver havido contratação específica para esse fim).

A Secretaria de Saúde (SMS) vai aumentar o repasse a um Organização
Social (OS) para que um novo prédio seja reformado e uma nova Unidade
Básica de Saúde (UBS) entre em funcionamento: deve-se contabilizar o
valor excedente repassado a OS.

A Secretaria de Desenvolvimento Econômico e Trabalho (SMDET) pactuou
novo contrato de gestão com a Agência São Paulo de Desenvolvimento
(ADESAMPA) em que um dos compromissos do Contrato de Gestão é uma ação
do Programa de Metas: deve-se contabilizar a proporção do Contrato de
Gestão equivalente às entregas do Programa de Metas.

\end{tcolorbox}

Não são consideradas despesas do Programa de Metas:

\begin{itemize}
\item
  Despesas de obras não previstas no Programa de Metas;
\item
  Despesas de custeio de equipamentos, programas ou serviços
  implementados em período anterior ao Programa de Metas vigente;
\item
  Despesas administrativas, cujo escopo contratado não tenha relação com
  metas ou ações previstas no Programa de Metas
\end{itemize}

\begin{tcolorbox}[enhanced jigsaw, titlerule=0mm, rightrule=.15mm, toprule=.15mm, breakable, leftrule=.75mm, left=2mm, colbacktitle=quarto-callout-tip-color!10!white, colframe=quarto-callout-tip-color-frame, toptitle=1mm, bottomtitle=1mm, coltitle=black, title=\textcolor{quarto-callout-tip-color}{\faLightbulb}\hspace{0.5em}{Tip}, colback=white, opacitybacktitle=0.6, arc=.35mm, bottomrule=.15mm, opacityback=0]

\textbf{A ``regra de ouro'' do monitoramento orçamentário do Programa de
Metas é capturar:}

\begin{itemize}
\item
  \textbf{os novos investimentos e}
\item
  \textbf{a expansão do custeio.}
\end{itemize}

\end{tcolorbox}

\bookmarksetup{startatroot}

\hypertarget{quais-suxe3o-as-etapas-de-monitoramento-oruxe7amentuxe1rio-do-programa-de-metas}{%
\chapter{Quais são as etapas de monitoramento orçamentário do Programa
de
Metas?}\label{quais-suxe3o-as-etapas-de-monitoramento-oruxe7amentuxe1rio-do-programa-de-metas}}

O monitoramento orçamentário do Programa de Metas segue um ciclo
composto por quatro fases principais: (1) Previsão de Custo; (2)
Orçamento Planejado; (3) Execução Orçamentária; (4) Avaliação. Cada
estágio envolve etapas de análise, validação e consolidação das
informações fornecidas pelos órgãos, de forma a acompanhar a execução
das metas e subsidiar a gestão dos recursos.

\hypertarget{previsuxe3o-de-custo}{%
\section{Previsão de custo}\label{previsuxe3o-de-custo}}

A fase de Previsão marca o início do ciclo e consiste em informar a
estimativa de custo quadrienal, segmentada por ano, para cada meta.

\begin{longtable}[]{@{}
  >{\raggedright\arraybackslash}p{(\columnwidth - 2\tabcolsep) * \real{0.1124}}
  >{\raggedright\arraybackslash}p{(\columnwidth - 2\tabcolsep) * \real{0.8876}}@{}}
\toprule\noalign{}
\endhead
\bottomrule\noalign{}
\endlastfoot
\textbf{Aspecto} & \textbf{Descrição} \\
\textbf{Quando ocorre} & Até 15 de Abril de cada exercício \\
\textbf{Quem faz} & Secretarias \\
\textbf{Onde inserir} & SMAE \textgreater{} Visão orçamentária
\textgreater{} Previsão de Custo \\
\textbf{O que inserir} & O custo estimado da Meta ou Ação, ano a ano,
até totalizar o quadriênio, informando partes da dotação orçamentária e
vinculando a dotação à Meta ou à Ação. \\
\end{longtable}

\textbf{Passos para preenchimento no SMAE:}

\begin{enumerate}
\def\labelenumi{\arabic{enumi}.}
\tightlist
\item
  Acesse a aba ``Programa de Metas''
\item
  No menu lateral esquerdo, clique em ``Metas''
\item
  Selecione a meta desejada
\item
  No menu lateral, clique em ``Previsão de custo''
\end{enumerate}

\begin{figure}

\includegraphics{figura1_tela_previsao_custo.png} \hfill{}

\end{figure}

\textbf{Figura 1.} Tela de Previsão de Custo no SMAE.

Na tela de \textbf{Previsão de Custo}, clique em ``\textbf{(+) ADICIONAR
PREVISÃO DE CUSTO}. É nesta tela que você informará:

\begin{itemize}
\tightlist
\item
  dotação orçamentária (parcial): Órgão, Unidade, Função,
  Projeto/Atividade e Fonte. Os demais são opcionais;
\item
  vinculação da dotação com a Meta OU com a ação estratégica;
\item
  Custo estimado da meta ou da ação
\end{itemize}

\begin{tcolorbox}[enhanced jigsaw, titlerule=0mm, rightrule=.15mm, toprule=.15mm, breakable, leftrule=.75mm, left=2mm, colbacktitle=quarto-callout-note-color!10!white, colframe=quarto-callout-note-color-frame, toptitle=1mm, bottomtitle=1mm, coltitle=black, title=\textcolor{quarto-callout-note-color}{\faInfo}\hspace{0.5em}{Note}, colback=white, opacitybacktitle=0.6, arc=.35mm, bottomrule=.15mm, opacityback=0]

O custo estimado da meta ou da ação é o valor monetário real ou estimado
para a sua concretização, independentemente dos valores orçados e
disponíveis nas dotações informadas.

\end{tcolorbox}

\begin{figure}

\includegraphics{figura2_adicionar_previsao_custo.png} \hfill{}

\end{figure}

\textbf{Figura 2.} Adicionando Previsão de Custo no SMAE.

\begin{tcolorbox}[enhanced jigsaw, titlerule=0mm, rightrule=.15mm, toprule=.15mm, breakable, leftrule=.75mm, left=2mm, colbacktitle=quarto-callout-note-color!10!white, colframe=quarto-callout-note-color-frame, toptitle=1mm, bottomtitle=1mm, coltitle=black, title=\textcolor{quarto-callout-note-color}{\faInfo}\hspace{0.5em}{Note}, colback=white, opacitybacktitle=0.6, arc=.35mm, bottomrule=.15mm, opacityback=0]

\begin{itemize}
\item
  Embora a inserção da Previsão de Custo deva ser concluído no SMAE até
  abril de cada exercício, é possível que SEPLAN/SIME/CPMA colete, em
  fevereiro e março, informações adicionais junto à sua pasta, em
  complemento ao que estiver disponível no SMAE, para fins de construção
  do \textbf{ANEXO DE METAS E INVESTIMENTOS}, do Projeto de Lei de
  Diretrizes Orçamentárias.
\item
  Esta coleta adicional \textbf{não se confunde com o processo conduzido
  por SEPLAN/COPOM para a elaboração do PLDO}, cujo prazo de entrega ao
  Legislativo é 15 de abril, de cada ano e que, por isso, tende a
  demandar informações de caráter geral (e não restritas ao Programa de
  Metas) nos períodos de fevereiro e março.
\item
  Note que a previsão de custo é mais intensiva no primeiro ano de
  vigência do Programa de Metas. Nos anos seguintes, podem ocorrer
  revisões. É plausível, inclusive, que a previsão de custo seja
  inserida uma única vez no quadriênio.
\item
  Diversos fatores podem ensejar necessidade de revisão de custos:

  \begin{itemize}
  \item
    Mudanças no modelo de implementação do equipamento, programa ou
    serviço (de contratação direta para PPP, por exemplo);
  \item
    Mudanças em variáveis macroeconômicas, preços de commodities e
    reequlíbrios econômico-financeiros de contratos;
  \item
    Mudanças no escopo, prazos ou itens de qualidade previamente
    especificados para o equipamento, programa ou serivço;
  \item
    Mudanças de fornecedores.
  \end{itemize}
\item
  É fundamental disponilizar para SEPLAN/SIME/CPMA \textbf{memórias de
  cálculo} sobre o custo da meta. Combine com o(a) ponto-focal o meio
  para entrega (SMAE, e-mail), mas não deixe de apresentar a composição
  do custo do equipamento, programa ou serviço pactuado no Programa de
  Metas.
\item
  \textbf{As informações sobre custos não fazem parte do monitoramento
  externo} do Programa de Metas, ou seja, não são disponibilizadas
  publicamente nos Relatórios de Execução Anual e nos balanços
  semestrais.
\end{itemize}

\end{tcolorbox}

\hypertarget{oruxe7amento-planejado}{%
\section{Orçamento planejado}\label{oruxe7amento-planejado}}

O registro do orçamento planejado no SMAE consiste em informar os
valores efetivamente contemplados na Lei Orçamentária de cada ano, nas
dotações vinculadas ao Programa de Metas.

\begin{longtable}[]{@{}
  >{\raggedright\arraybackslash}p{(\columnwidth - 2\tabcolsep) * \real{0.1156}}
  >{\raggedright\arraybackslash}p{(\columnwidth - 2\tabcolsep) * \real{0.8844}}@{}}
\toprule\noalign{}
\endhead
\bottomrule\noalign{}
\endlastfoot
\textbf{Aspecto} & \textbf{Descrição} \\
\textbf{Quando ocorre} & Até 15 de Abril de cada exercício \\
\textbf{Quem faz} & Secretarias \\
\textbf{Onde inserir} & SMAE \textgreater{} Visão orçamentária
\textgreater{} Orçamento planejado \\
\textbf{O que inserir} & Valor destinado ao Programa de Metas para o
exercício, no limite dos valores orçados aprovados na Lei de Orçamento
Anaual, vinculnado-o à Meta ou Ação \\
\end{longtable}

\textbf{Passos para preenchimento no SMAE:}

\begin{enumerate}
\def\labelenumi{\arabic{enumi}.}
\tightlist
\item
  Acesse a aba ``Programa de Metas''
\item
  No menu lateral esquerdo, clique em ``Metas''
\item
  Selecione a meta desejada
\item
  No menu lateral, clique em ``Orçamento Planejado''
\end{enumerate}

\begin{figure}

\includegraphics{figura3_tela_orcamento_planejado.png} \hfill{}

\end{figure}

\textbf{Figura 3.} Tela de Orçamento Planejado no SMAE.

Na tela de \textbf{Orçamento Planejado}, clique em ``\textbf{(+)
ADICIONAR DOTAÇÃO}. É nesta tela que você informará:

\begin{itemize}
\tightlist
\item
  dotação orçamentária completa;
\item
  vinculação da dotação com a Meta OU com a ação estratégica;
\item
  Valor da meta ou da ação para o exercício, limitado ao valor aprovado
  na LOA.
\end{itemize}

\begin{figure}

\includegraphics{figura4_adicionar_orcamento_planejado.png} \hfill{}

\end{figure}

\textbf{Figura 4.} Adicionando Orçamento Planejado no SMAE.

\begin{tcolorbox}[enhanced jigsaw, titlerule=0mm, rightrule=.15mm, toprule=.15mm, breakable, leftrule=.75mm, left=2mm, colbacktitle=quarto-callout-note-color!10!white, colframe=quarto-callout-note-color-frame, toptitle=1mm, bottomtitle=1mm, coltitle=black, title=\textcolor{quarto-callout-note-color}{\faInfo}\hspace{0.5em}{Note}, colback=white, opacitybacktitle=0.6, arc=.35mm, bottomrule=.15mm, opacityback=0]

Note que se as informações sobre a \textbf{PREVISÃO DE CUSTO} e sobre o
\textbf{ORÇAMENTO PLANEJADO} estiverem adequadamente registradas no
SMAE, tanto a sua pasta quanto SEPLAN conseguem mapear eventuais
\textbf{PRESSÕES ORÇAMENTÁRIAS} para cada exercício. Em outras palavras,
caso saibamos que se em uma determinada dotação, o custo previsto for
maior do que o valor orçado, é provável que a sua pasta necessitará de
mais recursos para concretização da Meta ou Ação vinculada a esta
dotação.

\end{tcolorbox}

\hypertarget{execuuxe7uxe3o-oruxe7amentuxe1ria}{%
\section{Execução
orçamentária}\label{execuuxe7uxe3o-oruxe7amentuxe1ria}}

O registro da execução orçamentária no SMAE consiste em informar os
valores efetivamente empenhados e liquidados com despesas do Programa de
Metas.

\begin{longtable}[]{@{}
  >{\raggedright\arraybackslash}p{(\columnwidth - 2\tabcolsep) * \real{0.0448}}
  >{\raggedright\arraybackslash}p{(\columnwidth - 2\tabcolsep) * \real{0.9552}}@{}}
\toprule\noalign{}
\endhead
\bottomrule\noalign{}
\endlastfoot
\textbf{Aspecto} & \textbf{Descrição} \\
\textbf{Quando ocorre} & \begin{minipage}[t]{\linewidth}\raggedright
\begin{itemize}
\item
  \textbf{Até 15 de Janeiro}: referente à execução orçamentária
  acumulada do \textbf{exercício anterior};
\item
  \textbf{Até 15 de Abril}: referente à execução acumulada dos
  \textbf{três primeiros meses do exercício};
\item
  \textbf{Até 15 de Julho}: referente à execução acumulada dos
  \textbf{seis primeiros meses do exercício};
\item
  \textbf{Até 15 de Outubro}: referente à execução acumulada dos
  \textbf{noves primeiros meses do exercício}.
\end{itemize}
\end{minipage} \\
\textbf{Quem faz} & Secretarias \\
\textbf{Onde inserir} & SMAE \textgreater{} Visão orçamentária
\textgreater{} Execução orçamentária \\
\textbf{O que inserir} & \begin{minipage}[t]{\linewidth}\raggedright
Despesas contabilizadas para fins de monitoramento orçamentário do
Programa de Metas, apuradas em:

\begin{itemize}
\item
  \textbf{dotações orçamentárias}: deve ser utilizada apenas quando
  existir uma \textbf{dotação orçamentária exclusiva} para o
  equipamento, programa ou serviço pactuado no Programa de Metas, de
  forma que todas as despesas que oneram a referida dotação sejam
  passíveis de contabilização para fins de monitoramento orçamentário do
  PdM.
\item
  \textbf{processos SEI}: deve ser utilizado quando não há dotação
  exclusiva, mas a despesa do Programa de Metas ocorreu por meio de uma
  \textbf{contratação exclusiva} para a entrega do equipamento, programa
  ou serviço pactuado no Programa de Metas. Nesses casos, deve-se
  informar os números dos processos de contratação (processos com Notas
  de Empenho) e os processos de pagamentos (processos com Notas de
  Liquidação e Pagamento).
\item
  \textbf{notas de empenho}: devem ser utilizadas quando parte de uma
  contratação é destinada aos compromissos do Programa de Metas ou
  quando o período de uma contratação exclusiva extrapola um exercício.
  Trata-se do nível de maior granularidade para o monitoramento
  orçamentário e, por esta razão, é \textbf{preferível em relação às
  demais opções}. \textbar{}
\end{itemize}
\end{minipage} \\
\end{longtable}

\textbf{Passos para preenchimento no SMAE:}

\begin{enumerate}
\def\labelenumi{\arabic{enumi}.}
\tightlist
\item
  Acesse a aba ``Programa de Metas''
\item
  No menu lateral esquerdo, clique em ``Metas''
\item
  Selecione a meta desejada
\item
  No menu lateral, clique em ``Execução Orçamentária''
\end{enumerate}

\begin{figure}

\includegraphics{figura5_tela_execucao_orcamentaria.png} \hfill{}

\end{figure}

\textbf{Figura 5.} Tela de Execução Orçamentária no SMAE.

Após analisar o atributo de \textbf{exclusividade} das dotações e
contratações, opte se a execução orçamentária acumulada do período será
informada por \textbf{dotação, processo ou nota de empenho}.

\hypertarget{dotauxe7uxe3o}{%
\subsection{Dotação}\label{dotauxe7uxe3o}}

\begin{figure}

\includegraphics{figura6_adicionar_execucao_dotacao.png} \hfill{}

\end{figure}

\textbf{Figura 6.} Adicionando Execução Orçamentária no SMAE, por
dotação.

Após fazer a vinculação da dotação orçamentária com a meta ou com a ação
estratégica:

\begin{enumerate}
\def\labelenumi{\arabic{enumi}.}
\item
  Clique em ``\textbf{(+) INFORMAR EXECUÇÃO ORÇAMENTÁRIA}''.
\item
  Informe o \textbf{mês de referência}: trata-se do mês até o qual os
  valores executados estão acumulados.
\item
  Informe o \textbf{Valor Executado}: trata-se do valor acumulado até o
  mês de referência. É possível obter o valor executado tanto pela
  inserção direta do valor quanto pela percentagem (0 a 100) referente
  ao Programa de Metas na dotação orçamentária informada.
\item
  Clique em \textbf{Salvar} para finalizar.
\end{enumerate}

\begin{figure}

\includegraphics{figura7_adicionar_execucao_dotacao_complemento.png} \hfill{}

\end{figure}

\textbf{Figura 7.} Adicionando Execução Orçamentária no SMAE, por
dotação - inserindo valores.

\hypertarget{processo}{%
\subsection{Processo}\label{processo}}

\begin{figure}

\includegraphics{figura8_adicionar_execucao_processo.png} \hfill{}

\end{figure}

\textbf{Figura 8.} Adicionando Execução Orçamentária no SMAE, por
processo

Após selecionar a dotação orçamentária e fazer a vinculação com a meta
ou com a ação estratégica, sigo os passos da \textbf{Figura 7}:

\begin{enumerate}
\def\labelenumi{\arabic{enumi}.}
\item
  Clique em ``\textbf{(+) INFORMAR EXECUÇÃO ORÇAMENTÁRIA}''.
\item
  Informe o \textbf{mês de referência}: trata-se do mês até o qual os
  valores executados estão acumulados.
\item
  Informe o \textbf{Valor Executado}: trata-se do valor acumulado até o
  mês de referência. É possível obter o valor executado tanto pela
  inserção direta do valor quanto pela percentagem (0 a 100) referente
  ao Programa de Metas na dotação orçamentária informada.
\item
  Clique em \textbf{Salvar} para finalizar.
\end{enumerate}

\hypertarget{nota-de-empenho}{%
\subsection{Nota de empenho}\label{nota-de-empenho}}

\begin{figure}

\includegraphics{figura9_adicionar_execucao_nota_empenho.png} \hfill{}

\end{figure}

\textbf{Figura 9.} Adicionando Execução Orçamentária no SMAE, por nota
de empenho

Após informar o número e o ano da nota de empenho, sigo os passos da
\textbf{Figura 7}:

\begin{enumerate}
\def\labelenumi{\arabic{enumi}.}
\item
  Clique em ``\textbf{(+) INFORMAR EXECUÇÃO ORÇAMENTÁRIA}''.
\item
  Informe o \textbf{mês de referência}: trata-se do mês até o qual os
  valores executados estão acumulados.
\item
  Informe o \textbf{Valor Executado}: trata-se do valor acumulado até o
  mês de referência. É possível obter o valor executado tanto pela
  inserção direta do valor quanto pela percentagem (0 a 100) referente
  ao Programa de Metas na dotação orçamentária informada.
\item
  Clique em \textbf{Salvar} para finalizar.
\end{enumerate}

\hypertarget{upload-de-arquivo-.xlsx}{%
\subsection{Upload de arquivo .xlsx}\label{upload-de-arquivo-.xlsx}}

Se deseja informar os dados de execução orçmentária via planilha:

\begin{enumerate}
\def\labelenumi{\arabic{enumi}.}
\item
  Acesse a aba ``Programa de Metas'';
\item
  No menu lateral esquerdo, clique em ``Envio de arquivos''.
\end{enumerate}

\begin{figure}

\includegraphics{figura10_tela_envio_arquivos.png} \hfill{}

\end{figure}

\textbf{Figura 10.} Tela para informar execução orçamentária via upload
de arquivo .xlsx

\begin{enumerate}
\def\labelenumi{\arabic{enumi}.}
\setcounter{enumi}{2}
\item
  Selecione a versão do PdM que está em uso;
\item
  Escolha a opção ``enviar arquivo'', localizada no canto superior
  direito.
\end{enumerate}

\begin{figure}

\includegraphics{figura11_tela_envio_arquivos_complemento.png} \hfill{}

\end{figure}

\textbf{Figura 11.} Tela para informar execução orçamentária via upload
de arquivo .xlsx

A planilha a ser inserida deve seguir o seguinte modelo:

\begin{figure}

\includegraphics[width=1\textwidth,height=\textheight]{figura12_modelo_planilha_upload.png} \hfill{}

\end{figure}

\textbf{Figura 12.} Modelo de planilha para informar execução
orçamentária via upload de arquivo .xlsx

Durante o preenchimento, alguns padrões precisam ser seguidos para que o
SMAE processe os dados:

\begin{itemize}
\item
  É necessário selecionar apenas uma forma de identificação das
  informações a serem inseridas, ou seja, dotação, processo SEI ou nota
  de empenho. Caso uma mesma linha da planilha contenha mais de um
  desses campos (por exemplo, o número da dotação e da nota de empenho
  simultaneamente), o SMAE não processará os dados corretamente,
  exigindo nova submissão da planilha ou ajuste manual das informações
\item
  No caso dos processos SEI, se o órgão optar por essa forma de
  identificação, é necessário informar, junto ao número do processo, a
  dotação orçamentária, pois pode haver mais de uma dotação vinculada a
  um mesmo processo. Caso essa especificação não seja incluída, o SMAE
  não processará a informação
\item
  As metas sempre devem ser informadas com três dígitos, como 002, 076
  ou 114.
\end{itemize}

\hypertarget{avaliauxe7uxe3o}{%
\section{Avaliação}\label{avaliauxe7uxe3o}}

A fase de Avaliação é realiada por \textbf{SEPLAN/SIME/CPMA}, com a
consolidação dos dados obtidos nas etapas anteriores, permitindo à
SEPLAN analisar o desempenho orçamentário e identificar riscos que
possam comprometer o atingimento das metas, como inexecução de recursos,
contingenciamentos ou descompasso entre gastos e entregas. A partir
dessa análise, são elaboradas recomendações e orientações técnicas
voltadas ao aperfeiçoamento da execução orçamentária e à mitigação de
riscos.

\bookmarksetup{startatroot}

\hypertarget{normas-aplicuxe1veis-ao-monitoramento-oruxe7amentuxe1rio}{%
\chapter{Normas aplicáveis ao monitoramento
orçamentário}\label{normas-aplicuxe1veis-ao-monitoramento-oruxe7amentuxe1rio}}

\hypertarget{programa-de-metas}{%
\section{Programa de Metas}\label{programa-de-metas}}

\href{https://legislacao.prefeitura.sp.gov.br/leis/decreto-64341-de-2-de-julho-de-2025}{Decreto
63.336/2024}: Estabelece procedimentos para o monitoramento e a
avaliação do Programa de Metas, previsto no artigo 69-A da Lei Orgânica
do Município de São Paulo; institui a Rede do Sistema de Monitoramento e
Acompanhamento Estratégico do Programa de Metas - Rede SMAE, dentre
outras providências.

Portaria SEPLAN/XX/2026: Dispõe sobre detalhamentos das orientações e
procedimentos atinentes ao monitoramento do Programa de Metas, previsto
no Art. 69-A da Lei Orgânica do Município de São Paulo, regulamentado
pelo Decreto nº 63.336, de 11 de abril de 2024

\hypertarget{execuuxe7uxe3o-oruxe7amentuxe1ria-1}{%
\section{Execução
orçamentária}\label{execuuxe7uxe3o-oruxe7amentuxe1ria-1}}

\href{http://legislacao.prefeitura.sp.gov.br/leis/decreto-64755-de-27-de-novembro-de-2025}{Decreto
64.755/2025}: Dispõe sobre o encerramento do exercício de 2025.

\href{https://legislacao.prefeitura.sp.gov.br/leis/decreto-64008-de-16-de-janeiro-de-2025}{Decreto
64.008/2025}: Fixa normas referentes à execução orçamentária e
financeira para o exercício de 2025.



\end{document}
